%&latex
\documentclass{report}

%+Make Index
\usepackage{makeidx}
\makeindex
%-Make Index

\begin{document}

%+Title
\title{\Huge\bf Export of Modelica models to the ProMoVis environment}
\author{Jesper Moberg}
\date{\today}
\maketitle
%-Title

%+Contents
\tableofcontents
%-Contents

\chapter{Templates}
There is text in first chapter

\section{About Template}
This template provides a sample layout of a Standard \LaTeX{} Report.

The front matter has a number of sample entries that you should replace
with your own. 

\section{Document Class Options}
The typesetting specification selected by this document template
uses the default class options. There are a number of class options 
supported by this document class. The available options include 
setting the paper size, the point size of the font used in the 
document body and others.

\subsection{Customizing Class Options}
Select `Insert', `Document Properties ...', the `Generic' tab
and then modify desired class options in appeared dialog.
Changes will be applied after pressing the 'OK' button.
\chapter {Description of JModelica}
\section{Background}
JModelica is an open-source Modelica environment, written in python,  for compilation and simulation of Modelica models. Through its python front-end it provides an easy to use, still powerful way to perform complex tasks on compiled models. 

\section{Example model}
When explaining we sometimes reference this model to clarify stuff, fixme
%% Document title: QuadTankPack.mo

\input highlight.sty

{
\tt
{\hlstd }{\hlkwa package\ }{\hlstd QuadTankPack\leavevmode\par
\leavevmode\par
}{\hlstd\ \ }{\hlstd }{\hlkwa model\ }{\hlstd Sim\_{}QuadTank\leavevmode\par
}{\hlstd\ \ \ \ }{\hlstd QuadTank\ qt}{\hlopt ;}\leavevmode\par
{\hlstd }{\hlstd\ \ \ \ }{\hlstd }{\hlkwa input\ }{\hlstd Real\ u1\ }{\hlopt $\mathord{=}$\ }{\hlstd qt.u1}{\hlopt ;}\leavevmode\par
{\hlstd }{\hlstd\ \ \ \ }{\hlstd }{\hlkwa input\ }{\hlstd Real\ u2\ }{\hlopt $\mathord{=}$\ }{\hlstd qt.u2}{\hlopt ;}\leavevmode\par
{\hlstd }{\hlstd\ \ }{\hlstd }{\hlkwb initial\ }{\hlstd }{\hlkwa equation}\leavevmode\par
{\hlstd }{\hlstd\ \ }{\hlstd }{\hlslc //der(qt.x1)\ $\mathord{=}$\ 0;}\leavevmode\par
{\hlstd }{\hlstd\ \ }{\hlstd }{\hlslc //der(qt.x2)\ $\mathord{=}$\ 0;}\leavevmode\par
{\hlstd }{\hlstd\ \ \ \ }{\hlstd qt.x1\ }{\hlopt $\mathord{=}$\ }{\hlstd }{\hlnum 0.0627}{\hlstd }{\hlopt ;}\leavevmode\par
{\hlstd }{\hlstd\ \ \ \ }{\hlstd qt.x2\ }{\hlopt $\mathord{=}$\ }{\hlstd }{\hlnum 0.06044}{\hlstd }{\hlopt ;}\leavevmode\par
{\hlstd }{\hlstd\ \ \ \ }{\hlstd qt.x3\ }{\hlopt $\mathord{=}$\ }{\hlstd }{\hlnum 0.024}{\hlstd }{\hlopt ;}\leavevmode\par
{\hlstd }{\hlstd\ \ \ \ }{\hlstd qt.x4\ }{\hlopt $\mathord{=}$\ }{\hlstd }{\hlnum 0.023}{\hlstd }{\hlopt ;}\leavevmode\par
{\hlstd }{\hlstd\ \ }{\hlstd }{\hlkwa end\ }{\hlstd Sim\_{}QuadTank}{\hlopt ;}\leavevmode\par
{\hlstd \leavevmode\par
}{\hlstd\ \ }{\hlstd }{\hlkwa model\ }{\hlstd QuadTank\leavevmode\par
}{\hlstd\ \ \ \ }{\hlstd }{\hlslc //\ Process\ parameters}\leavevmode\par
{\hlstd \ }{\hlkwa parameter\ }{\hlstd Modelica.SIunits.Area\ A1}{\hlopt $\mathord{=}$}{\hlstd }{\hlnum 4.9e$\mathord{-}$4}{\hlstd }{\hlopt ,\ }{\hlstd A2}{\hlopt $\mathord{=}$}{\hlstd }{\hlnum 4.9e$\mathord{-}$4}{\hlstd }{\hlopt ,\ }{\hlstd A3}{\hlopt $\mathord{=}$}{\hlstd }{\hlnum 4.9e$\mathord{-}$4}{\hlstd }{\hlopt ,\ }{\hlstd A4}{\hlopt $\mathord{=}$}{\hlstd }{\hlnum 4.9e$\mathord{-}$4}{\hlstd }{\hlopt ;}\leavevmode\par
{\hlstd \ }{\hlkwa parameter\ }{\hlstd Modelica.SIunits.Area\ a1}{\hlopt (}{\hlstd min}{\hlopt $\mathord{=}$}{\hlstd }{\hlnum 1e$\mathord{-}$6}{\hlstd }{\hlopt )$\mathord{=}$}{\hlstd }{\hlnum 0.03e$\mathord{-}$4}{\hlstd }{\hlopt ,\ }{\hlstd a2}{\hlopt $\mathord{=}$}{\hlstd }{\hlnum 0.03e$\mathord{-}$4}{\hlstd }{\hlopt ,\ }{\hlstd a3}{\hlopt $\mathord{=}$}{\hlstd }{\hlnum 0.03e$\mathord{-}$4}{\hlstd }{\hlopt ,\ }{\hlstd a4}{\hlopt $\mathord{=}$}{\hlstd }{\hlnum 0.03e$\mathord{-}$4}{\hlstd }{\hlopt ;}\leavevmode\par
{\hlstd \ }{\hlkwa parameter\ }{\hlstd Modelica.SIunits.Acceleration\ g}{\hlopt $\mathord{=}$}{\hlstd }{\hlnum 9.81}{\hlstd }{\hlopt ;}\leavevmode\par
{\hlstd \ }{\hlkwa parameter\ }{\hlstd Real\ k1\_{}nmp}{\hlopt (}{\hlstd unit}{\hlopt $\mathord{=}$}{\hlstd }{\hlstr "m{\bf\^{}}3/s/V"}{\hlstd }{\hlopt )\ $\mathord{=}$\ }{\hlstd }{\hlnum 0.56e$\mathord{-}$6}{\hlstd }{\hlopt ,\ }{\hlstd k2\_{}nmp}{\hlopt (}{\hlstd unit}{\hlopt $\mathord{=}$}{\hlstd }{\hlstr "m{\bf\^{}}3/s/V"}{\hlstd }{\hlopt )\ $\mathord{=}$\ }{\hlstd }{\hlnum 0.56e$\mathord{-}$6}{\hlstd }{\hlopt ;}\leavevmode\par
{\hlstd \ }{\hlkwa parameter\ }{\hlstd Real\ g1\_{}nmp}{\hlopt $\mathord{=}$}{\hlstd }{\hlnum 0.30}{\hlstd }{\hlopt ,\ }{\hlstd g2\_{}nmp}{\hlopt $\mathord{=}$}{\hlstd }{\hlnum 0.30}{\hlstd }{\hlopt ;}\leavevmode\par
{\hlstd \leavevmode\par
}{\hlstd\ \ \ \ }{\hlstd }{\hlslc //\ Initial\ tank\ levels}\leavevmode\par
{\hlstd \ }{\hlkwa parameter\ }{\hlstd Modelica.SIunits.Length\ x1\_{}0\ }{\hlopt $\mathord{=}$\ }{\hlstd }{\hlnum 0.04102638}{\hlstd }{\hlopt ;}\leavevmode\par
{\hlstd \ }{\hlkwa parameter\ }{\hlstd Modelica.SIunits.Length\ x2\_{}0\ }{\hlopt $\mathord{=}$\ }{\hlstd }{\hlnum 0.06607553}{\hlstd }{\hlopt ;}\leavevmode\par
{\hlstd \ }{\hlkwa parameter\ }{\hlstd Modelica.SIunits.Length\ x3\_{}0\ }{\hlopt $\mathord{=}$\ }{\hlstd }{\hlnum 0.00393984}{\hlstd }{\hlopt ;}\leavevmode\par
{\hlstd \ }{\hlkwa parameter\ }{\hlstd Modelica.SIunits.Length\ x4\_{}0\ }{\hlopt $\mathord{=}$\ }{\hlstd }{\hlnum 0.00556818}{\hlstd }{\hlopt ;}\leavevmode\par
{\hlstd \ \leavevmode\par
}{\hlstd\ \ \ \ }{\hlstd }{\hlslc //\ Tank\ levels}\leavevmode\par
{\hlstd \ Modelica.SIunits.Length\ x1}{\hlopt (}{\hlstd start}{\hlopt $\mathord{=}$}{\hlstd x1\_{}0}{\hlopt ,}{\hlstd min}{\hlopt $\mathord{=}$}{\hlstd }{\hlnum 0.0001}{\hlstd }{\hlcom /*,max$\mathord{=}$0.20*/}{\hlstd }{\hlopt );}\leavevmode\par
{\hlstd \ Modelica.SIunits.Length\ x2}{\hlopt (}{\hlstd start}{\hlopt $\mathord{=}$}{\hlstd x2\_{}0}{\hlopt ,}{\hlstd min}{\hlopt $\mathord{=}$}{\hlstd }{\hlnum 0.0001}{\hlstd }{\hlcom /*,max$\mathord{=}$0.20*/}{\hlstd }{\hlopt );}\leavevmode\par
{\hlstd \ Modelica.SIunits.Length\ x3}{\hlopt (}{\hlstd start}{\hlopt $\mathord{=}$}{\hlstd x3\_{}0}{\hlopt ,}{\hlstd min}{\hlopt $\mathord{=}$}{\hlstd }{\hlnum 0.0001}{\hlstd }{\hlcom /*,max$\mathord{=}$0.20*/}{\hlstd }{\hlopt );}\leavevmode\par
{\hlstd \ Modelica.SIunits.Length\ x4}{\hlopt (}{\hlstd start}{\hlopt $\mathord{=}$}{\hlstd x4\_{}0}{\hlopt ,}{\hlstd min}{\hlopt $\mathord{=}$}{\hlstd }{\hlnum 0.0001}{\hlstd }{\hlcom /*,max$\mathord{=}$0.20*/}{\hlstd }{\hlopt );}\leavevmode\par
{\hlstd \leavevmode\par
\ }{\hlslc //\ Inputs}\leavevmode\par
{\hlstd \ }{\hlkwa input\ }{\hlstd Modelica.SIunits.Voltage\ u1}{\hlopt ;}\leavevmode\par
{\hlstd \ }{\hlkwa input\ }{\hlstd Modelica.SIunits.Voltage\ u2}{\hlopt ;}\leavevmode\par
{\hlstd \leavevmode\par
}{\hlstd\ \ }{\hlstd }{\hlkwa equation}\leavevmode\par
{\hlstd }{\hlstd\ \ \ \ }{\hlstd }{\hlkwb der}{\hlstd }{\hlopt (}{\hlstd x1}{\hlopt )\ $\mathord{=}$\ $\mathord{-}$}{\hlstd a1}{\hlopt /}{\hlstd A1}{\hlopt *}{\hlstd sqrt}{\hlopt (}{\hlstd }{\hlnum 2}{\hlstd }{\hlopt *}{\hlstd g}{\hlopt *}{\hlstd x1}{\hlopt )\ $\mathord{+}$\ }{\hlstd a3}{\hlopt /}{\hlstd A1}{\hlopt *}{\hlstd sqrt}{\hlopt (}{\hlstd }{\hlnum 2}{\hlstd }{\hlopt *}{\hlstd g}{\hlopt *}{\hlstd x3}{\hlopt )\ $\mathord{+}$}\leavevmode\par
{\hlstd }{\hlstd\ \ \ \ \ }{\hlstd g1\_{}nmp}{\hlopt *}{\hlstd k1\_{}nmp}{\hlopt /}{\hlstd A1}{\hlopt *}{\hlstd u1}{\hlopt ;}\leavevmode\par
{\hlstd \ }{\hlkwb der}{\hlstd }{\hlopt (}{\hlstd x2}{\hlopt )\ $\mathord{=}$\ $\mathord{-}$}{\hlstd a2}{\hlopt /}{\hlstd A2}{\hlopt *}{\hlstd sqrt}{\hlopt (}{\hlstd }{\hlnum 2}{\hlstd }{\hlopt *}{\hlstd g}{\hlopt *}{\hlstd x2}{\hlopt )\ $\mathord{+}$\ }{\hlstd a4}{\hlopt /}{\hlstd A2}{\hlopt *}{\hlstd sqrt}{\hlopt (}{\hlstd }{\hlnum 2}{\hlstd }{\hlopt *}{\hlstd g}{\hlopt *}{\hlstd x4}{\hlopt )\ $\mathord{+}$}\leavevmode\par
{\hlstd }{\hlstd\ \ \ \ \ }{\hlstd g2\_{}nmp}{\hlopt *}{\hlstd k2\_{}nmp}{\hlopt /}{\hlstd A2}{\hlopt *}{\hlstd u2}{\hlopt ;}\leavevmode\par
{\hlstd \ }{\hlkwb der}{\hlstd }{\hlopt (}{\hlstd x3}{\hlopt )\ $\mathord{=}$\ $\mathord{-}$}{\hlstd a3}{\hlopt /}{\hlstd A3}{\hlopt *}{\hlstd sqrt}{\hlopt (}{\hlstd }{\hlnum 2}{\hlstd }{\hlopt *}{\hlstd g}{\hlopt *}{\hlstd x3}{\hlopt )\ $\mathord{+}$\ (}{\hlstd }{\hlnum 1}{\hlstd }{\hlopt $\mathord{-}$}{\hlstd g2\_{}nmp}{\hlopt )*}{\hlstd k2\_{}nmp}{\hlopt /}{\hlstd A3}{\hlopt *}{\hlstd u2}{\hlopt ;}\leavevmode\par
{\hlstd \ }{\hlkwb der}{\hlstd }{\hlopt (}{\hlstd x4}{\hlopt )\ $\mathord{=}$\ $\mathord{-}$}{\hlstd a4}{\hlopt /}{\hlstd A4}{\hlopt *}{\hlstd sqrt}{\hlopt (}{\hlstd }{\hlnum 2}{\hlstd }{\hlopt *}{\hlstd g}{\hlopt *}{\hlstd x4}{\hlopt )\ $\mathord{+}$\ (}{\hlstd }{\hlnum 1}{\hlstd }{\hlopt $\mathord{-}$}{\hlstd g1\_{}nmp}{\hlopt )*}{\hlstd k1\_{}nmp}{\hlopt /}{\hlstd A4}{\hlopt *}{\hlstd u1}{\hlopt ;}\leavevmode\par
{\hlstd \leavevmode\par
}{\hlstd\ \ }{\hlstd }{\hlkwa end\ }{\hlstd QuadTank}{\hlopt ;}\leavevmode\par
{\hlstd }\leavevmode\par
{\hlkwa end\ }{\hlstd QuadTankPack}{\hlopt ;}{\hlstd }\leavevmode\par
}
\bye
% TeX generated by Highlight 3.8, http://www.andre-simon.de/


\section{How the export tool uses JModelica}
JModelica is used to compile the Modelica models to JModelicas JMU representation \cite{PythonAPI}. This representation is internally represented as a, possibly, nonlinear DAE. This model, can through the JModelica environemnt be linearized and a model with the following represenation can be extracted:
\begin{equation}
E*dx = A*x + B*u + F*w + g
\end{equation}
The current version of JModelica\cite{JModelicaVer} does not separate between internal states and outputs. Naturally, the x and dx vectors represents the states and outputs of the linearized system. The u vector represents declared inputs, w is modeled disturbances (FIXME are they modeled in modelica, or attached at simulation time cant find anything regarding this in modelica specification) and finally g is a constant bias.\\The linearization also outputs some useful information that we later use in the generation of ProMoVis scenarios:
\begin{itemize}
\item \textit{State names}, corresponding to the declared variable names from the original Modelica file.
\item \textit{Input names}, corresponding to the declared input names from the original Modelica file.
\item \textit{Working points} for the linear model \textit{dx0}, \textit{u0} and \textit{x0}. Which is used to provide feedback for the user regarding the (FIXME)sanity of the linearized model.
\end{itemize}

The export tool makes some assumptions about the 
%+Bibliography
\begin{thebibliography}{99}
\bibitem{PythonAPI} JModelica.org python api docs 
\bibitem{JModelicaVer} 1.7b2
\end{thebibliography}
%-Bibliography

%+MakeIndex
\printindex
%-MakeIndex


\end{document}


