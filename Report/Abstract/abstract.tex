ProMoVis provides an environment where engineers may, through a combination of graphical visualisation and powerful analysis tools, analyse and deepen their understanding of complex processes and its components. Currently a drawback is that the user has to create a model of the system, through the graphical interface that ProMoVis provides, even though he may have existing models, created in another environment.\\\newline This report describes the development process and implementation of a tool, written in python, that utilizes the JModelica environment to export Models, described in Modelica, to the equivalent SFG representation used in ProMoVis. This is done through linearising the models around an operating point and then extracting a DAE representation of it. From this DAE, it is then possible to extract an SFG representation, describing how a variable depends on other variables in the system.\\\newline Besides demonstrating the key algorithm for extracting the SFGs this report also discusses the design of the actual software and how the internals of the tools is built up into separate phases. Each responsible for compiling the original model, building an intermediate representation for SFGs used internally, validation and some optional transformations of the original model and finally the actual generation of the ProMoVis representation from the intermediate representation. 