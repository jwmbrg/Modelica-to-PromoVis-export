ProMoVis provides an environment where engineers may, through a combination of graphical visualisation and powerful analysis tools, analyse and deepen their understanding of complex processes and its components. A drawback is that a users has to manually derive the information needed for the complex process to be modeled and anlysed in ProMoVis. A task that, in most cases, where one has non-linear and large processses may become very tedious. At the same time users may have existing models, created in another environment, that already contains much of the information needed for the analysis in ProMoVis.\\\newline This report describes the design process and implementation of a tool, utilizing JModelica to linearise and import existing Modelica models into ProMoVis. By making the user supply information regarding the operating points of the states in the system, one can extract a Differential Algebraic Equation(DAE) as a representation of the linear model. From this DAE it is then possibly to build the Signal Flow Graph(SFG) representation used in ProMoVis.\\\newline Besides demonstrating the key algorithm for extracting the SFGs this report also discusses the design of the software and how the internal structure of the tool has been divided into separate phases to accomodate for the different types of subproblems ecountered in the export process. 

%Responsible for compiling the original model, building an intermediate representation of the SFGs for internal use, validation of the system together with some optional transformations of the original model and finally the actual generation of the ProMoVis representation from the intermediate representation. 