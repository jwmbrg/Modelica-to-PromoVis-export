\section{Workflow}
During the projects initial phase much of the time was put on familiarizing with the ProMoVis and JModelica enviroments and the math that acts as a foundation for the Modelica models. During these first weeks there were sporadic meetings and discussions with the supervisor whenever problems or questions appeared. After an initial meeting with the developer of the ProMoVis frontend (See Appendix \ref{appC}). The start of the development of a first "Proof of concept" was initiated. After finishing this, a more thorough design phase started where both the developers of ProMoVis and a "test-user" was involved. 
\section{Tools}
\subsection{Modelica}
Modelica is a object-oriented declarative programming language used to create models of physical system. Unlike many other modelling languages, Modelica is not domain specific and thus can be used to model physical systems consisting of a mix of electrical, mechanical and chemical processes. Since the release of the first language specification in 1997 the continuous development of the language has been maintained by the Modelica Association and the current version of the Language specification is 3.2 \cite{ModelicaSpec}.\nocite{*}
\subsection{JModelica}
The initial demand on the environment to be used was that is should be Open-source. After examining the options available JModelica\cite{jmodelicaorg}\nocite{*} was choosen because of its Python front-end, making it easy to interface with the tool from other software, while still providing a structured environment around which we can build all parts of the export tool.   
\subsection{Numpy and SciPy}
Another important feature supporting the use of the JModelica platform was its tight integration with the NumPy and SciPy packages \cite{scipyorg}\nocite{*}. These are two very popular Open source packages for Python providing a strong and well documented toolbox for mathematical computing. When extracting the mathematical models through JModelica they are represented using the data-structures provided by NumPy and in many cases this removes the need to "reinvent the wheel" for many of the operations that needs to be performed on the mathematical models before export. Instead one can directly use existing methods provided by SciPy and NumPy. The use of this package also makes the source-code for the export tool more accesible to other programmers familiar with Python and these packages.
\subsection{FIXME IS THERE A NEED TO Review THE CONCEPTS OF THE SFG }
And how it is used in promovis.
