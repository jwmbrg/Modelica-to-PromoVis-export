\section{Background}
%During a research project at LTU a tool was developed, called ProMoVis,  aimed at structure analysis and visualisation of complex physical processes.
While Modelica, Simulink and other existing tools are very powerful when it comes to simulation and modelling of  dynamic systems there is a need for a tool that can aid engineers in selecting different control structures by making it possible to analyse specific interconnections in dynamic systems. %A drawback with existing tools is that although a user declares the relations between inputs and outputs, at simulation time the models are still assumed to be treated as "black boxes". Making it hard to monitor and analyse the internal behaviour of components in the modelled systems. 
Due to the identified need for a tool that can solve these problems \cite{ProMoVisPaper}\nocite{*}, the design and implementation of ProMoVis started in 2010 which recently was released as an open source project.
\section{Problem Description}
As stated earlier, ProMoVis purpose is visualisation and analysis of large systems. It assumes that the user has information about the components of the system and the relations between them.  Today, %there is no scripting interface to ProMoVis and 
models have to be created in the graphical environment by placing the components and defining the relations one by one. This is a task that can become very tedious and error-prone for many systems. At the same time a user that would benefit from using ProMoVis in his design process can be assumed to already have existing models, from another environment, available. Therefore the possibility to import existing models would make ProMoVis more accessible to its intended users.
\section{Project Goal}
In recent time Modelica, as a tool for modeling of physical systems, has shown an increase in popularity and has become somewhat of a standard for describing physical systems. Due to its widespread use and the amount of open source tools that are available for models described with help from the language, this projects goal is to implement a tool that can import models, described in Modelica, into ProMoVis. This tool should try to preserve as much information and semantics from the original Modelica models but, at the same time, keep the effort needed from the user to perform the export at a minimum. A final demand is that the tool should be easy to integrate with the existing ProMoVis environment so that imports can be performed from within ProMoVis.
% Features needed
%There is a need to a tool for control structure selection. It should analyse the interconnections in a dynamic system and how a control strategy should be shaped. This very different from what you write here.]



