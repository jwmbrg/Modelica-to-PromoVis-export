\section{Discussion}
\subsection{The project}
\subsubsection{Problems}
The project has over time progressed slowly but surely. A big challenge has been the lack of experience and knowledge of the mathematical tools used in modelling of physical systems. The problems have still been managable, with the help from Wolfgang and Miguel, but i should have, in the beginning of the project, planned more time for excersising and reading in to them. Now, instead, i used much of the time reading in to specifications, papers and tools related to Modelica. And even though Modelica is a big part, it is in the implementation of the tool, a very small part and how we extract the information from Modelica is very tool dependent.\\\newline Another problem, as i can see it, is that one should have tried to get in touch with someone experienced with the JModelica environment specifically. JModelica consists of many parts, written in different languages and utilizing many 3rd party tools, and sometimes it has been kind of messy to navigate through the documentation and define where the parts, relevant to this project, might reside. So a lot of time probably could have been saved if there had been a possibility to pitch som questions and design decisions to a developer or experienced user of JModelica.\\\newline These two problems together have led to many dead-ends and misunderstandings on my part, leading to a lot of additional work due to bugs and faulty assumptions about the format of the extracted data.
\subsection{Limitations of the tool}
\subsubsection{Additional variable info}
A thing that should be implemented is the ability to extract min, max, nominal and other additional parameters that a variable may have in a Modelica-model since this information would also be useful in a generated ProMoVis-representation. But right now there does not seem to be a way to extract this through JModelica. A forum post has been created, at jmodelica.org, regarding the absence of this possibility.
\subsubsection{The graphical representation}
Although the translation between Modelica and the ProMoVis-representation is now possible, the two environments is not integrated with each others in such a way that one can do modifications in Modelica, and then only the affected parts of the system is changed in ProMoVis. This is a significant drawback when it comes to the visualisation purpose of ProMoVis. A user that has imported a Modelica model into ProMoVis and done some custom layout and maybe added additional graphical representation through ProMoVis components, can't reuse this information when re-running an export, even though he might only have changed the operating points of the Modelica model.\\\newline There exists two general approaches to this problem: 
\begin{itemize}
\item One could add the possibility to do the export with a former ProMoVis-file as additional input. Then let the export tool could scan through the old file, look for variables with the same name and re-use the graphical information from that node. This approach still demands that the user add the graphical information needed for visualisation after the first export.
\item Another approach is to add support for extracting information from graphical annotations. The graphical annotations are a standardised way, through the specification of the language \cite{ModelicaSpec}\nocite{*},of creating a visual representation of a modeled system and its components.
\end{itemize}Of this two approaches, the later should be the one to aim for since, if there is time and resources available, supporting the interpretation of graphical annotations in the export-tool could make the transition between a users Modelica environment and ProMoVis a one-click effort. Since the tool then might preserve the graphical context from the Modelica environment.
\subsubsection{Elimination of unwanted variables}
As of today, all variables present in the original Modelica model is present in the generated ProMoVis-representation. Paths in the SFG containing only internal variables could instead be eliminated and collected into one single transfer function during the transformation phase. The challenge here is to handle loops in such a path so that the resulting transfer function is correct. The intention was to support this before the project ended, but due to lack of time this was never implemented. 

\section{Closing remarks}
This project has shown that it is indeed possible to extract and translate models from Modelica into ProMoVis. For future development the challenge does not lie in any issues related to control theory or mathematical challenges, the theory needed for this already exists and "just" needs to be implemented. Instead, the challenge lies in making it convenient for the user to use the export tool, and make it as seamless as possible to transition from the Modelica environment to the ProMoVis environment. 
