%&latex
\documentclass{report}

%+Make Index
\usepackage{makeidx}
\makeindex
%-Make Index

\begin{document}

%+Title
\title{\Huge\bf Export of Modelica models to the ProMoVis environment}
\author{Jesper Moberg}
\date{\today}
\maketitle
%-Title

%+Contents
\tableofcontents
%-Contents

\chapter{Templates}
There is text in first chapter

\section{About Template}
This template provides a sample layout of a Standard \LaTeX{} Report.

The front matter has a number of sample entries that you should replace
with your own. 

\section{Document Class Options}
The typesetting specification selected by this document template
uses the default class options. There are a number of class options 
supported by this document class. The available options include 
setting the paper size, the point size of the font used in the 
document body and others.

\subsection{Customizing Class Options}
Select `Insert', `Document Properties ...', the `Generic' tab
and then modify desired class options in appeared dialog.
Changes will be applied after pressing the 'OK' button.
\chapter {Description of JModelica}
\section{Background}
JModelica is an open-source Modelica environment, written in python,  for compilation and simulation of Modelica models. Through its python front-end it provides an easy to use, still powerful way to perform complex tasks on compiled models. 

\section{How the export tool uses JModelica}
JModelica is used to compile the Modelica models to JModelicas JMU representation \cite{PythonAPI}. This representation is internally represented as a, possibly, nonlinear DAE. This model, can through the JModelica environemnt be l�inearized and a model with the following represenation can be extracted:
\begin{equation}
E*dx = A*x + B*u + F*w + g
\end{equation}
Naturally, x
%+Bibliography
\begin{thebibliography}{99}
\bibitem{PythonAPI} JModelica.org python api docs
\bibitem{Label2} ...
\end{thebibliography}
%-Bibliography

%+MakeIndex
\printindex
%-MakeIndex


\end{document}


